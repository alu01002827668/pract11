\documentclass{beamer}
\usepackage[utf8]{inputenc}
\usepackage{graphicx}

\newtheorem{definicion}{Definición}
\newtheorem{ejemplo}{Ejemplo}

%%%%%%%%%%%%%%%%%%%%%%%%%%%%%%%%%%%%%%%%%%%%%%%%%%%%%%%%%%%%%%%%%%%%%%%%%%%%%%%
\title[Número $\pi$]{Práctica 11 }
\author[Yessica]{Yessica Sabrina Gómez Buso}
\date[25-04-2014]{25 de abril de 2014}
%%%%%%%%%%%%%%%%%%%%%%%%%%%%%%%%%%%%%%%%%%%%%%%%%%%%%%%%%%%%%%%%%%%%%%%%%%%%%%%

%\usetheme{Madrid}
%\usetheme{Antibes}
%\usetheme{tree}
%\usetheme{classic}

%%%%%%%%%%%%%%%%%%%%%%%%%%%%%%%%%%%%%%%%%%%%%%%%%%%%%%%%%%%%%%%%%%%%%%%%%%%%%%%
\begin{document}
  
%++++++++++++++++++++++++++++++++++++++++++++++++++++++++++++++++++++++++++++++  
\begin{frame}

  %\includegraphics[width=0.15\textwidth]{img/ullesc}
  \hspace*{7.0cm}
  %\includegraphics[width=0.16\textwidth]{img/fmatesc}
  \titlepage

  \begin{small}
    \begin{center}
     Facultad de Matemáticas \\
     Universidad de La Laguna
    \end{center}
  \end{small}

\end{frame}
%++++++++++++++++++++++++++++++++++++++++++++++++++++++++++++++++++++++++++++++  

%++++++++++++++++++++++++++++++++++++++++++++++++++++++++++++++++++++++++++++++  
\begin{frame}
  \frametitle{Índice}  
  \tableofcontents[pausesections]
\end{frame}
%++++++++++++++++++++++++++++++++++++++++++++++++++++++++++++++++++++++++++++++  


\section{Primera Sección}


%++++++++++++++++++++++++++++++++++++++++++++++++++++++++++++++++++++++++++++++  
\begin{frame}

\frametitle{Primera Sección}

\begin{definicion}
 El número $\pi$ es la relación entre la longitud de la circunferencia y su diámetro.
\end{definicion}

\end{frame}
%++++++++++++++++++++++++++++++++++++++++++++++++++++++++++++++++++++++++++++++  

\section{Segunda Sección}

%++++++++++++++++++++++++++++++++++++++++++++++++++++++++++++++++++++++++++++++  
\begin{frame}

\frametitle{Segunda Sección}

\begin{block}{Fórmulas}
  \begin{itemize}
  \item
  Primer item 
  \pause

  \item
  Segundo item
  \pause

  \item
  Tercer item

  \end{itemize}
\end{block}

\end{frame}
%++++++++++++++++++++++++++++++++++++++++++++++++++++++++++++++++++++++++++++++  

\section{Ejercicios}

\subsection{Una subsección}
%++++++++++++++++++++++++++++++++++++++++++++++++++++++++++++++++++++++++++++++  
\begin{frame}
\frametitle{El numero $\pi$}

Texto de la diapositiva
\end{frame}
%++++++++++++++++++++++++++++++++++++++++++++++++++++++++++++++++++++++++++++++  

\subsection{Creación de diapositivas}

%++++++++++++++++++++++++++++++++++++++++++++++++++++++++++++++++++++++++++++++  
\begin{frame}
\frametitle{Diapositivas}

\begin{definition}
  Un ejemplo de definición
\end{definition}

\begin{example}
  \begin{itemize}
    \item <1-> Primero \pause
    \item <2-> Segundo \pause
    \item <3-> Tercero \pause
    \item <4-> Cuarto  
  \end{itemize}
\end{example}

\end{frame}
%++++++++++++++++++++++++++++++++++++++++++++++++++++++++++++++++++++++++++++++  

\subsection{Otra subseccion}
%++++++++++++++++++++++++++++++++++++++++++++++++++++++++++++++++++++++++++++++  
\begin{frame}
\frametitle{Este es otro Título}

\begin{definicion}
  Otra definición 
\end{definicion}

\begin{ejemplo}
  \begin{enumerate}
    \item
      Primero
      \pause

    \item
      Segundo 

  \end{enumerate}
\end{ejemplo}

\end{frame}
%++++++++++++++++++++++++++++++++++++++++++++++++++++++++++++++++++++++++++++++  

\section{Bibliografía}
%++++++++++++++++++++++++++++++++++++++++++++++++++++++++++++++++++++++++++++++  
\begin{frame}
  \frametitle{Bibliografía}

  \begin{thebibliography}{10}

    \beamertemplatebookbibitems
    \bibitem[Plan Estudios, 2011]{plan}  
    Documento de verificación del grado. 
    (2011) 

    \beamertemplatebookbibitems
    \bibitem[Guía Docente, 2013]{guia}  
    Guía docente. 
    (2013) 
    {\small $http://eguia.ull.es/matematicas/query.php?codigo=299341201$}

    \beamertemplatebookbibitems
    \bibitem[URL: CTAN]{latex} 
    CTAN. {\small $http://www.ctan.org/$}

  \end{thebibliography}
\end{frame}

%++++++++++++++++++++++++++++++++++++++++++++++++++++++++++++++++++++++++++++++  
\end{document}
